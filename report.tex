%% BioMed_Central_Tex_Template_v1.06
%%                                      %
%  bmc_article.tex            ver: 1.06 %
%                                       %

%%IMPORTANT: do not delete the first line of this template
%%It must be present to enable the BMC Submission system to
%%recognise this template!!

%%%%%%%%%%%%%%%%%%%%%%%%%%%%%%%%%%%%%%%%%
%%                                     %%
%%  LaTeX template for BioMed Central  %%
%%     journal article submissions     %%
%%                                     %%
%%          <8 June 2012>              %%
%%                                     %%
%%                                     %%
%%%%%%%%%%%%%%%%%%%%%%%%%%%%%%%%%%%%%%%%%


%%%%%%%%%%%%%%%%%%%%%%%%%%%%%%%%%%%%%%%%%%%%%%%%%%%%%%%%%%%%%%%%%%%%%
%%                                                                 %%
%% For instructions on how to fill out this Tex template           %%
%% document please refer to Readme.html and the instructions for   %%
%% authors page on the biomed central website                      %%
%% http://www.biomedcentral.com/info/authors/                      %%
%%                                                                 %%
%% Please do not use \input{...} to include other tex files.       %%
%% Submit your LaTeX manuscript as one .tex document.              %%
%%                                                                 %%
%% All additional figures and files should be attached             %%
%% separately and not embedded in the \TeX\ document itself.       %%
%%                                                                 %%
%% BioMed Central currently use the MikTex distribution of         %%
%% TeX for Windows) of TeX and LaTeX.  This is available from      %%
%% http://www.miktex.org                                           %%
%%                                                                 %%
%%%%%%%%%%%%%%%%%%%%%%%%%%%%%%%%%%%%%%%%%%%%%%%%%%%%%%%%%%%%%%%%%%%%%

%%% additional documentclass options:
%  [doublespacing]
%  [linenumbers]   - put the line numbers on margins

%%% loading packages, author definitions

\documentclass[twocolumn]{bmcart}% uncomment this for twocolumn layout and comment line below
%\documentclass{bmcart}

%%% Load packages
%\usepackage{amsthm,amsmath}
%\RequirePackage{natbib}
%\RequirePackage{hyperref}
\usepackage[hidelinks]{hyperref}
\usepackage[utf8]{inputenc} %unicode support
\usepackage{xcolor}
\hypersetup{
    colorlinks,
    linkcolor={red!50!black},
    citecolor={blue!50!black},
    urlcolor={blue!80!black}
}
%\usepackage[applemac]{inputenc} %applemac support if unicode package fails
%\usepackage[latin1]{inputenc} %UNIX support if unicode package fails
\usepackage{graphicx}

%%%%%%%%%%%%%%%%%%%%%%%%%%%%%%%%%%%%%%%%%%%%%%%%%
%%                                             %%
%%  If you wish to display your graphics for   %%
%%  your own use using includegraphic or       %%
%%  includegraphics, then comment out the      %%
%%  following two lines of code.               %%
%%  NB: These line *must* be included when     %%
%%  submitting to BMC.                         %%
%%  All figure files must be submitted as      %%
%%  separate graphics through the BMC          %%
%%  submission process, not included in the    %%
%%  submitted article.                         %%
%%                                             %%
%%%%%%%%%%%%%%%%%%%%%%%%%%%%%%%%%%%%%%%%%%%%%%%%%


%\def\includegraphic{figures/Features.pdf}
%\def\includegraphics{}



%%% Put your definitions there:
\startlocaldefs
\endlocaldefs


%%% Begin ...
\begin{document}

%%% Start of article front matter
\begin{frontmatter}

\begin{fmbox}
\dochead{GDP Individual Report}

%%%%%%%%%%%%%%%%%%%%%%%%%%%%%%%%%%%%%%%%%%%%%%
%%                                          %%
%% Enter the title of your article here     %%
%%                                          %%
%%%%%%%%%%%%%%%%%%%%%%%%%%%%%%%%%%%%%%%%%%%%%%

\title{What is Important for Technology Companies to Make Money?}

%%%%%%%%%%%%%%%%%%%%%%%%%%%%%%%%%%%%%%%%%%%%%%
%%                                          %%
%% Enter the authors here                   %%
%%                                          %%
%% Specify information, if available,       %%
%% in the form:                             %%
%%   <key>={<id1>,<id2>}                    %%
%%   <key>=                                 %%
%% Comment or delete the keys which are     %%
%% not used. Repeat \author command as much %%
%% as required.                             %%
%%                                          %%
%%%%%%%%%%%%%%%%%%%%%%%%%%%%%%%%%%%%%%%%%%%%%%

\author[
   email={sh3g12@soton.ac.uk}   % email address
]{\fnm{Shakib-Bin} \snm{Hamid}}

%\end{fmbox}% comment this for two column layout

%%%%%%%%%%%%%%%%%%%%%%%%%%%%%%%%%%%%%%%%%%%%%%
%%                                          %%
%% The Abstract begins here                 %%
%%                                          %%
%% Please refer to the Instructions for     %%
%% authors on http://www.biomedcentral.com  %%
%% and include the section headings         %%
%% accordingly for your article type.       %%
%%                                          %%
%%%%%%%%%%%%%%%%%%%%%%%%%%%%%%%%%%%%%%%%%%%%%%

\begin{abstractbox}

\begin{abstract} % abstract
Gene regulatory \cite{winterman-kelly-online-shopper} is used to perform hybrid modelling of GRN.
\end{abstract}

\end{abstractbox}
%
\end{fmbox}% uncomment this for twcolumn layout

\end{frontmatter}

%%%%%%%%%%%%%%%%%%%%%%%%% start of article main body
% <put your article body there>

%%%%%%%%%%%%%%%%
\section*{The `Idea'}
A technology company starts with an idea, but the idea alone is just part of the puzzle and carries less weight in revenues, profit margins or even valuation than some other concepts like market, protection, growth etc. During a guest lecture Alexander Hill from Senseye put this crucial point accross, as did David Parker.\\

\par Selling videos has long been a successful business. In fact Blockbuster and HMV were largely successful in doing exactly that. But these companies are now dead \cite{gavinjackson2013} or skeletons of their former selves \cite{markwembridgeclaerbarrett2011}. On the other hand, Netflix and Apple took the same idea with a different delivery medium - streaming content. Youtube is also incredibly successful with the same idea of streaming video, and yet I do not consider Youtube and Netflix to be competitors, since they serve very different purposes - Netflix as a paid service for produced videos, whereas Youtube is an open platform of community content. Given the market dominance of Netflix among its targeted users, one would not expect Amazon, an online retailer, to be a worthy opponent. Yet Amazon Instant video, a business spinned from the acquisition of LoveFilm \cite{timbradshaw2011}, is gearing up for intense fight this year \cite{madhumitamurgia2017} and they are already a competitor to Netflix. But Amazon's success is obviously not in the idea, but in the delivery of its content - through its Prime subscription framework - the Flywheel shown by Ian Gavin, i.e. their retail business ,which gets more traffic from the Instant Video customers, drives revenue. The two companies are operating with the same `idea', but they are both successful because of their business model. On the other hand, Apple, an extremely profitable company and an earlier adopter to the streaming idea than Amazon, is failing to make a measurable dent in the video streaming market, although their iTunes music store is a market leader. It is because of failing to secure content deals in time \cite{matthewgarrahan2017} and a very narrow delivery route - the iTunes ecosystem, which many associate with Apple devices and do not consider to be a standalone service. Similarly, Vine - a company with an exciting idea and bought by Twitter, is now in ashes \cite{chrisfoxx2016}. They could not make money from their `short vlog' idea in the long run, although it was quite popular among the users. Both Apple and Vine had the right `idea' but their execution has fallen short.\\

\par The point to make here is that the `idea' and acting early on it is surely important - but only if the other parts of the business model is executed well as done by Netflix. It is quite possible to enter a busy market with a largely similar idea and successfully compete with a majority market holder as done by Amazon Instant Video. But the idea alone will not make the business successful as demonstrated by Vine and Apple's iTunes video store.

\section*{The Market and Distribution Channel}
David Parker put market and delivery as the most important part of a technology business during his guest lecture, as did Ian Gavin. Finding the right audience and delivering the product to them in an efficient and customer-friendly manner is absolutely crucial to the success and survival of a technology company.\\

\par Amazon takes pride in their `One Click' to buy and as short as a two hour delivery service - Prime Now. Amazon's target has always been to work backwards from the customer and drive volume based on customer satisfaction. With that in mind, they are investing billions in huge fulfillment centres and Prime Air this year \cite{lesliehook2016}. Amazon's recent focus on taking control over the logistics is a proof that distribution channels are crucial to their ever growing share in the retail world. They have also realised that their market share in the US is reaching saturation and are now focusing on Middle East and Asia, evident in the Souq.com acquisition \cite{simeonkerr2017} and Prime launch in India \cite{simonmundy2016}. On the other hand, Google has had to create a market for their smartphone business. As the maintainer of the Android project, Google has let companies like Samsung, LG, Sony etc. infiltrate the smartphone market with Android users. Now that the market is ready, Google themselves have entered the market with their premium Pixel devices and marketed them to have the `best and pure' form of Android, quickly overtaking LG and Sony. However, Apple has taken a drastically different approach to both of these companies. Unlike Amazon that operates on volume and Google that operates largely on ads, Apple has successfully marketed themselves as the de-facto premium `gadget makers'. They market their product as premium and `it just works'. Thus Apple has been able to sell millions of expensive iPhones \cite{tessstynes2015}, when Amazon's relatively cheap Fire Phone is now a dead product \cite{jacobkastrenakes2015}, proving that only pricing low may not win the war, while right target audience will always buy the product if they perceive it necessary.\\

\par Clearly recognising the right market and generating enough volume through the right distribution channels is crucial for a technology company to make money. All three companies above focus on their core competences, create the market and dominate it through volume, ease of access and perceived quality.

\section*{Protection}

A technology company must be able to protect its Intellectual Property, e.g. the formula, logo, product design, manufacturing process etc. so that others are not able to take their spot or make competing products. There are some legal procedures - copyright, trademark, patents etc., as well as creating other barriers to competition to achieve this.\\

\par Using copyright and registered trademark a company is able to protect their designs. Companies protect their novel approaches to solving a problem using patents. However, as Benjamin Husband put during his guest lecture, all the above requires significant time and money. Specially, patents can take months and once approved they are open for public. Unless the company is ready to litigate infringement, it may be better to keep it a trade secret, e.g. the Coca-Cola formula. Also, a granted patent is only valid where the issuing authority operates. So, to operate in multiple markets around the world, one would require multiple patents and keep them alive. A patent is also sought after to protect own company. Google purchased Motorola in 2012 exactly for this purpose and then sold the stripped off company to Lenovo \cite{waltersbradshawhammond2014} so that they could protect Android. There have been numerous lengthy and expensive legal battles over patent infringements, e.g. Apple Inc. v. Samsung Electronics Co \cite{bradshaw2015}. It is clear that these successful companies value their IP in the highest regard. Some like ARM have in fact orchestrated their business model about lisencing their IP. For others like Blackberry and Nokia, they have been revived recently because of their IP \cite{thomas2014}. Using their protected IP, a company can leverage their relationship with others. For example: recently Imagination Technologies suffered a heavy blow from their lapsed deal with Apple \cite{vincent2017}. But their IP gives them a fighting chance to challenge Apple in near future. On the other hand, Non-Practicing Entities like Acacia Research purchases patents and regularly exercises their right to litigate or lisence.\\

\par The key idea is that, technology companies must protect their IP - through legal process or trade secret. Protection gives them barriers to entry, revenues through lisencing, leverage in exit negotiation or even revival as a new entity in case of calamity.
      
\section*{Conclusion}
Biological regulatory networks of any biological problem are very important to study and analyse. GenNet facilitates the user to completely simulate and analyse both qualitative and quantitative behaviours of GRNs.

%%%%%%%%%%%%%%%%%%%%%%%%%%%%%%%%%%%%%%%%%%%%%%
%%                                          %%
%% Backmatter begins here                   %%
%%                                          %%
%%%%%%%%%%%%%%%%%%%%%%%%%%%%%%%%%%%%%%%%%%%%%%

\begin{backmatter}

% if your bibliography is in bibtex format, use those commands:
\nocite{*}
\bibliographystyle{ieeetr} % Style BST file (bmc-mathphys, vancouver, spbasic).
\bibliography{refs}      % Bibliography file (usually '*.bib' )

\end{backmatter}
\end{document}
































